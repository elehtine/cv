\documentclass{cv}

\usepackage[utf8]{inputenc}
\usepackage{lmodern}
\usepackage[T1]{fontenc}
\usepackage[finnish]{babel}

\usepackage{xcolor}
\usepackage{hyperref}
\usepackage{paracol} 
\usepackage{lipsum} 

\usepackage[margin=.5cm]{geometry}

\pagestyle{empty}


\begin{document}


\columnratio{0.3}
\backgroundcolor{c[0](.5cm, .5cm)(0cm, \textheight)}[rgb]{.7,.7,.7}


\begin{paracol}{2}
  % SIDEBAR

  \framebox{\Huge{Image}}

  \separator

  \begin{tabular}{cl}
    x & 20.8.2001 \\
    x & 00560 Helsinki \\
    x & elias.lehtinen01@gmail.com \\
    x & 044 336 5547 \\
    x & \href{https://www.github.com/elehtine}{elehtine} \\
  \end{tabular}

  \separator

  \skilllang{Suomi}{Äidinkieli}
  \skilllang{Englanti}{B2}

  \separator

  \section*{Kuvaus teknologioista}

  \skillprogramming{Java}{Spring}
  \skillprogramming{C++}{Qt, Boost, CMake}
  \skillprogramming{Python}{Fast API, Flask, \\ numpy, scipy, pandas, matplotlib}
  \skillprogramming{JavaScript}{TypeScript, Node, \\ Express, React, Svelte}
  \skillprogramming{Golang}{Charmbracelet}

  \switchcolumn


  % MAIN

  \name{Elias Lehtinen}

  \finnish{
  \section*{Profiili}

  Olen neljännen vuoden tietojenkäsittelytieteen opiskelija. Pidän
  matemaattisesta ongelmanratkaisusta ja uusien asioiden opettelusta. Käytän
  vapaa-aikaani ohjelmointiprojektien tekemiseen ja kavereiden kanssa pelaamisen.
  Harrastan myös shakin pelaamista ja erilaisten Rubikin kuution kaltaisten
  pulmapelien ratkaisua.
  Joulukuussa tein Advent of Code ohjelmointihaasteista 46/50 tähteä.
}

\english{
  \section*{Profile}

  I am a fourth-year computer science student. I love mathematical
  problem-solving and learning new things. As a hobby, I make software projects
  and play with my friends. I also play chess and solve different variations of
  Rubik's Cubes.
  In last December I got 46/50 stars in Advent of Code.
}


  \section*{Työkokemus}
  \event{Varusmiespalvelus}{01/2022 -- 12/2022}{
    Suoritin asepalveluksen erityistehtävässä
    virtuaalikoulutustukihenkilönä. Olin mukana ohjelmistoprojekteissa ja
    organisoin toimintaa valtakunnallisesti.
  }
  \separator
  \event{Muisoft}{06/2020 -- 12/2021}{
    Kavereiden kanssa perustettu startup.
  }
  \separator
  \event{PSIL}{08/2017 -- 05/2019}{
    Lukion ohessa järjestetty työharjoittelu. Olin mukana
    ohjelmistoprojekteissa ja sain vastuutehtäviä.
  }

  \section*{Opinnot}
  \event{Tietojenkäsittelytieteen kandiohjelma}{08/2019 alkaen}{
    Opiskelen tietojenkäsittelytiedettä Helsingin yliopistossa. Valmistun
    kandiksi seuraavana keväänä. Kandidaatin tutkielman aiheena oli
    Voronoikaavioiden soveltaminen laskennallisen geometrian ongelmissa. Olen
    tehnyt kursseja tilastotieteestä ja koneoppimisesta.
  }
  \separator
  \event{Päivölän matematiikkalinja}{06/2017 -- 05/2019}{
    Suoritin lukion Päivölän matematiikkalinjalla. Siellä lukio suoritetaan
    kahdessa vuodessa ja voitimme koulujoukkueiden SM-kisat molempina vuosina.
    Sain Teknologiateollisuuden 100-vuotissäätiöltä 1\,000 euron stipendin
    erinomaisesti suoritetusta pitkän matematiikan kokeesta.
  }

\end{paracol}

\end{document}

