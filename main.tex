\documentclass{cv}

\usepackage[utf8]{inputenc}
\usepackage{lmodern}
\usepackage[T1]{fontenc}
\usepackage[finnish]{babel}

\usepackage{xcolor}
\usepackage{hyperref}
\usepackage{paracol} 
\usepackage{lipsum} 

\usepackage[margin=.5cm]{geometry}

\pagestyle{empty}


\begin{document}


\columnratio{0.3}
\backgroundcolor{c[0](.5cm, .5cm)(0cm, \textheight)}[rgb]{.7,.7,.7}


\begin{paracol}{2}
  % SIDEBAR

  \framebox{\Huge{Image}}

  \sidesep

  \begin{tabular}{cl}
    x & 20.8.2001 \\
    x & Helsinki \\
    x & elias.lehtinen01@gmail.com \\
    x & 044 336 5547 \\
    x & \href{https://www.github.com/elehtine}{elehtine} \\
  \end{tabular}

  \sidesep

  \skilllang{Suomi}{Äidinkieli}
  \skilllang{Englanti}{B2}

  \switchcolumn


  % MAIN

  \name{Elias Lehtinen}

  \section*{Työkokemus}
  \begin{event}
    \eventitem{2022-2022}{Suomen Puolustusvoimat}{
      Suoritin asepalveluksen erityistehtävässä
      virtuaalikoulutustukihenkilönä. Olin mukana ohjelmointiprojekteissa ja
      organisoin toimintaa valtakunnallisesti.
    }
    \eventitem{2020-2021}{Muisoft}{
      Kavereiden kanssa perustettu startup.
    }
    \eventitem{2017-2019}{PSIL}{
      Lukion ohessa järjestetty työharjoittelu. Olin mukana
      ohjelmistoprojekteissa ja sain vastuutehtäviä.
    }
  \end{event}

  \section*{Opinnot}
  \begin{event}
    \eventitem{2019-2022}{Tietojenkäsittelytieteen kandiohjelma}{
      Opiskelen tietojenkäsittelytiedettä Helsingin yliopistossa. Valmistun
      kandiksi seuraavana keväänä.
    }
    \eventitem{2017-2019}{Päivölän matematiikkalinja}{
      Suoritin lukion Päivölän matematiikkalinjalla. Sain
      Teknologiateollisuuden 100-vuotissäätiöltä 1000 euron stipendin pitkän
      matematiikan ylioppilaskirjoituksen suorituksesta.
    }
  \end{event}

\end{paracol}

\end{document}

